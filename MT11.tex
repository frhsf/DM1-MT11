\documentclass[12pt,a4paper, french]{article} 
\usepackage[T1]{fontenc}              
\usepackage[utf8]{inputenc}  
\usepackage[french]{babel}
\frenchbsetup{StandardLists=true}
\usepackage{graphicx}
\usepackage{amsmath}
\usepackage[hidelinks]{hyperref}
\usepackage[left=2cm,right=2cm,top=2cm,bottom=2cm]{geometry}
\usepackage[skip=0.13cm]{parskip}
\usepackage{times}
\usepackage{amsmath}
\setlength\parindent{20pt}

\title{DM1 MT11}
\author{Léo Eugène, Mathieu Poveda, Sacha Hénaff}
\date{}
\begin{document}


\maketitle

\section*{Exercice 1}

1) On cherche de développement limité de cos(x) à l'ordre 4 en 0. On a:
\begin{center}
    \begin{equation*}
        cos(x)=cos(0)-xsin(0)-\frac{x^2}{2}cos(0)+\frac{x^2}{6}sin(0)+\frac{x^4}{24}cos(0)+x^4\epsilon(x) 
    \end{equation*}
\end{center}

Donc:
\begin{center}
    \begin{equation*} 
        cos(x)=1-\frac{x^2}{2}+\frac{x^4}{24}+x^4\epsilon(x)
    \end{equation*}
\end{center}

On cherche le développement limité de \begin{math}
    (1-cos(x))^2
\end{math}. On a: 
\begin{center}
    \begin{equation*}
        1-cos(x)=\frac{x^2}{2}-\frac{x^4}{24}+x^4\epsilon(x)
    \end{equation*}
\end{center}

De plus, on sait que le DL de \begin{math}
    x^2=x^2
\end{math}. Donc par composition, on a:
\begin{center}
    \begin{equation*}
        (1-cos(x))^2=(\frac{x^2}{2}-\frac{x^4}{24})^2+x^4\epsilon(x)
    \end{equation*}
\end{center}

En ne gardant que les termes d'ordres 4 maximum, on obtient:

\begin{center}
    \begin{equation*}
        (1-cos(x))^2=\frac{x^4}{4}+x^4\epsilon(x)
    \end{equation*}
\end{center}

On cherche le DL de \begin{math}
    cos(x)^2\end{math}. On a:

    \begin{center}
        \begin{equation*}
            cos(x)^2=(1-\frac{x^2}{2}+\frac{x^4}{24})^2+x^4\epsilon(x)
        \end{equation*}
    \end{center}

En développant cette expression et en ne gardant que les termes d'ordres 4 maximum, on a:
\begin{center}
    \begin{equation*}
        cos(x)^2=1-\frac{x^2}{2}+\frac{x^4}{24}-\frac{x^2}{2}+\frac{x^4}{4}+\frac{x^4}{24}+x^4\epsilon(x)
    \end{equation*}
\end{center}
\begin{center}
    \begin{equation*}
        cos(x)^2=1-x^2+\frac{x^4}{3}+x^4\epsilon(x)
    \end{equation*}
\end{center}

De plus, on a:
\begin{center}
    \begin{equation*}
        (1-cos(x)^2)=1-2cos(x)+cos(x)^2=1-(2-x^2+\frac{x^4}{12})+1-x^2+\frac{x^4}{4}+2\frac{x^4}{24}+x^4\epsilon(x)
    \end{equation*}
\end{center}

On obtient finalement que le DL de \begin{math}:(1-cos(x))^2\end{math} est:
\begin{center}
    \begin{equation*}
    (1-cos(x))^2=\frac{x^4}{4}+x^4\epsilon
    \end{equation*}
\end{center}

On retrouve bien le même résultat que précédemment.

2) a. On cherche à exprimer le développement de Taylor-Lagrange à l'ordre 4 au voisinnage de \begin{math}
    \frac{\pi}{4}
\end{math} de cos(x):
\begin{center}
    \begin{multline*}
        cos(x)=cos(\frac{\pi}{4})+(x-\frac{\pi}{4})cos'(\frac{\pi}{4})+\frac{(x-\frac{\pi}{4})^2}{2}cos''(\frac{\pi}{4}) \\
        +\frac{(x-\frac{\pi}{4})^3}{3!}cos'''(\frac{\pi}{4})+\frac{(x-\frac{\pi}{4})^4}{4!}cos''''(\frac{\pi}{4})+\frac{(x-\frac{\pi}{4})^5}{5!}cos'''''(\frac{\pi}{4}+\theta h)
    \end{multline*}
\end{center}

On a donc:
\begin{center}
    \begin{equation*}
        cos(x)=\frac{\sqrt{2}}{2}-(x-\frac{\pi}{4})sin(\frac{\pi}{4})-\frac{(x-\frac{\pi}{4})^2}{2}cos(\frac{\pi}{4})
        +\frac{(x-\frac{\pi}{4})^3}{3!}sin(\frac{\pi}{4})+\frac{(x-\frac{\pi}{4})^4}{4!}cos(\frac{\pi}{4})+\frac{(x-\frac{\pi}{4})^5}{5!}sin(\frac{\pi}{4}+\theta h)
    \end{equation*}
\end{center}

On a finalement:
\begin{center}
    \begin{equation*}
       cos(x)=\frac{\sqrt{2}}{2}-\frac{\sqrt{2}}{2}(x-\frac{\pi}{4})-\frac{\sqrt{2}}{4}(x-\frac{\pi}{4})^2+\frac{\sqrt{2}}{12}(x-\frac{\pi}{4})^3+\frac{\sqrt{2}}{48}(x-\frac{\pi}{4})^4+\frac{(x-\frac{\pi}{4})^5}{5!}sin(\frac{\pi}{4}+\theta h)
    \end{equation*}
\end{center}
On retrouve bien le même résultat que précédemment
  
b) On cherche le développement de Taylor-Young de cos(x) à l'ordre 4 en \begin{math}
    \frac{\pi}{4}
\end{math}. On, sait que les parties régulières des développement de Taylor-Young et Taylor-Lagrange sont les mêmes, seuls les restes varient. On a donc:

\begin{center}
    \begin{equation*}
       cos(x)=\frac{\sqrt{2}}{2}-\frac{\sqrt{2}}{2}(x-\frac{\pi}{4})-\frac{\sqrt{2}}{4}(x-\frac{\pi}{4})^2+\frac{\sqrt{2}}{12}(x-\frac{\pi}{4})^3+\frac{\sqrt{2}}{48}(x-\frac{\pi}{4})^4+(x-\frac{\pi}{4})^4\epsilon(x-\frac{\pi}{4})
    \end{equation*}
\end{center}

c) On réalise un changement de variable, on pose \begin{math}
    x=h+\frac{\pi}{4}
\end{math}. On sait que:
\begin{center}
    \begin{equation*}
       cos(a+b)=cos(a)cos(b)-sin(a)sin(b)
    \end{equation*}
\end{center}

Donc:
\begin{center}
    \begin{equation*}
       cos(x)=cos(h+\frac{\pi}{4})=cos(h)cos(\frac{\pi}{4})-sin(h)sin(\frac{\pi}{4})=\frac{\sqrt{2}}{2}cos(h)-\frac{\sqrt{2}}{2}sin(h)
    \end{equation*}
\end{center}

d)

On pose: \begin{math}
    x=h+\frac{\pi}{4} h=x-\frac{\pi}{4}
\end{math} 
On a: 
\begin{center}
    \begin{equation*}
       sin(h)=h-\frac{h^3}{3!}+h^4\epsilon(h)
    \end{equation*}
\end{center}
Et:
\begin{center}
    \begin{equation*}
       cos(h)=1-\frac{h^2}{2}+\frac{h^4}{24}+h^4\epsilon(h)
    \end{equation*}
\end{center}

On remplce h par son expression, on obtient:
\begin{center}
    \begin{equation*}
       sin(h)=(x-\frac{\pi}{4})-\frac{(x-\frac{\pi}{4})^3}{3!}+x^4\epsilon(x)
    \end{equation*}
\end{center}
Et:
\begin{center}
    \begin{equation*}
        cos(x)=1-\frac{(x-\frac{\pi}{4})^2}{2}+\frac{(x-\frac{\pi}{4})^4}{24}+x^4\epsilon(x)
    \end{equation*}
\end{center}

On remplace les expressions de sin et cos dans l'expression de la question c). On obtient:
\begin{center}
    \begin{equation*}
        cos(h+\frac{\pi}{4})=\frac{\sqrt{2}}{2}(1-\frac{(x-\frac{\pi}{4})^2}{2}+\frac{(x-\frac{\pi}{4})^4}{24}-(x-\frac{\pi}{4})-\frac{(x-\frac{\pi}{4})^3}{3!})+x^4\epsilon(x)
    \end{equation*}
\end{center}
En développant, on a:
\begin{center}
    \begin{equation*}
        cos(h+\frac{\pi}{4})=cos(x)=\frac{\sqrt{2}}{2}-\frac{\sqrt{2}}{2}(x-\frac{\pi}{4})-\frac{\sqrt{2}(x-\frac{\pi}{4})^2}{4}+\frac{\sqrt{2}(x-\frac{\pi}{4})^3}{24}+\frac{\sqrt{2}(x-\frac{\pi}{4})^4}{48}+(x-\frac{\pi}{4})^4\epsilon(x)
    \end{equation*}
\end{center}    
On retrouve bien l'expression de la question b.
\newpage
\section*{Exercice 2}
 On cherche \begin{math}
    \lim_{x \to 0} \frac{cos(x)}{sin(x)}
 \end{math}, qui n'est pas une forme indéterminée puis cos(x) tend vers 1 en 0 et sin(x) tend vers 0 en 0. On a donc: \begin{math}
    \lim_{x \to 0}\frac{cos(x)}{sin(x)}=+ \infty
 \end{math}.

 On cherche \begin{math}\lim_{x \to 0}\frac{1-cos(x)}{sin(x)}\end{math}. On connait les DL de cos(x) et sin(x), on peut donc aussi connaître le DL de 1-cos(x). On peut ainsi utiliser les parties principales des DL pour calculer la limite. On a:
 \begin{center}
    \begin{equation*}
        1-cos(x)=\frac{x^2}{2}-\frac{x^4}{24}+x^4\epsilon(x)
    \end{equation*}
\end{center}

Et: 
\begin{center}
    \begin{equation*}
        sin(x)=x-\frac{x^3}{6}+x^3\epsilon(x)
    \end{equation*}
\end{center}

Donc: 
\begin{center}
    \begin{equation*}
        \frac{1-cos(x)}{sin(x)}=\frac{\frac{x^2}{2}+x^2\epsilon(x)}{x+x\epsilon(x)}=\frac{x^2(\frac{1}{2}+\epsilon(x))}{x(1+\epsilon(x))}=x\frac{\frac{1}{2}+\epsilon(x)}{1+\epsilon(x)}
    \end{equation*}
\end{center}

Ainsi, on voit que:
\begin{center}
    \begin{equation*}
        \lim_{x \to 0}\frac{1-cos(x)}{sin(x)}=0
    \end{equation*}
\end{center}

On a:
\begin{center}
    \begin{math}
        ln(1+x)=x-\frac{x^2}{2}+x^2\epsilon(x)\end{math} et \begin{math}
            sin(x)+x=2x+x\epsilon(x)
        \end{math} 
\end{center}

Donc:
\begin{center}
    \begin{equation*}
    \frac{ln(1+x)}{sin(x)+x}=\frac{x+\epsilon(x)}{2x+\epsilon(x)}=\frac{x(1+\epsilon(x))}{x(2+\epsilon(x))}=\frac{1+\epsilon(x)}{2+\epsilon(x)}    
    \end{equation*}
\end{center}

Donc: 
\begin{center}
    \begin{equation*}
    lim_{x \to 0}\frac{ln(1+x)}{sin(x)+x}=\frac{1}{2}
    \end{equation*}
\end{center}

De même, on a: \begin{math}
    sin(x)+x^3=x+\frac{5x^3}{6}+x^3\epsilon(x)
\end{math}. Donc:
\begin{center}
    \begin{equation*}
    \frac{ln(1+x)}{sin(x)+x^3}=\frac{x+\epsilon(x)}{x+\epsilon(x)}=\frac{x(1+\epsilon(x))}{x(1+\epsilon(x))}=\frac{1+\epsilon(x)}{1+\epsilon(x)}
    \end{equation*}
\end{center}
\newpage
Donc:
\begin{center}
    \begin{equation*}
    lim_{x \to 0}\frac{ln(1+x)}{sin(x)+x^3}=1
    \end{equation*}
\end{center}

De même, on a: \begin{math}
    sin(x)-x=-\frac{x^3}{6}+x^3\epsilon(x)
\end{math}. Ainsi:
\begin{center}
    \begin{equation*}
       \frac{ln(1+x)}{sin(x)-x}=\frac{x+x\epsilon(x)}{-\frac{x^3}{6}+x^3\epsilon(x)}=\frac{x(1+\epsilon(x))}{x(-\frac{x^2}{6}+x^2\epsilon(x))}=-\frac{(1+\epsilon(x))}{(\frac{x^2}{6}+x^2\epsilon(x))}
    \end{equation*}
\end{center}

On a donc:
\begin{center}
    \begin{equation*}
       lim_{x \to 0}\frac{ln(1+x)}{sin(x)-x}=-\infty
    \end{equation*}
\end{center}

On sait que le DL de \begin{math}
    x^n=x^n
\end{math}. On a donc:
\begin{center}
    \begin{equation*}
    \frac{x^n}{sin(x)-x}=\frac{x+x\epsilon(x)}{-\frac{x^3}{6}+x^3\epsilon(x)}=x^n\frac{1}{x^3(-\frac{1}{6}+\epsilon(x))}=x^{n-3}\frac{1}{-\frac{1}{6}+\epsilon(x)}
    \end{equation*}
\end{center}

On doit donc distinguer 3 cas pour déterminer la limite de cette expression. Premièrement si n=3:
\begin{center}
    \begin{equation*}
    lim_{x \to 0}\frac{x^n}{sin(x)-x}=-6
    \end{equation*}
\end{center}

Si \begin{math}
    n<3
\end{math}, on a:
\begin{center}
    \begin{equation*}
    lim_{x \to 0}\frac{x^n}{sin(x)-x}=-\infty
    \end{equation*}
\end{center}

Enfin, si \begin{math}
    n>3
\end{math}, on a:
\begin{center}
    \begin{equation*}
    lim_{x \to 0}\frac{x^n}{sin(x)-x}=0
    \end{equation*}
\end{center}
\newpage

\section*{Exercice 3}

On a: \begin{math}f(x,y)=(x^2+y^2)^{3/2}
\end{math}

1) On cherche \begin{math}
    \frac{\partial^2f}{\partial x^2}
\end{math}

On a: 
\begin{center}
    \begin{equation*}
        \frac{\partial f}{\partial x}=\frac{6x}{2}(x^2+y^2)^{1/2}=3x(x^2+y^2)^{1/2}
    \end{equation*}
\end{center}

Et donc:
\begin{center}
    \begin{equation*}
        \frac{\partial^2f}{\partial x^2}=u'v+v'u=3(x^2+y^2)^{1/2}+3x^2(x^2+y^2)^{-1/2}=\frac{3(x^2+y^2)+3x^2}{(x^2+y^2)^{1/2}}=\frac{3(2x^2+y^2)}{(x^2+y^2)^{1/2}}
    \end{equation*}
\end{center}

2) On a: \begin{math}
    g(r,\theta)=f(rcos(\theta),rsin(\theta))
\end{math}

a. On a donc:
\begin{center}
    \begin{equation*}
        g(r,\theta)=f(rcos(\theta),rsin(\theta))=((rcos(\theta))^2+rsin(\theta)^2)^{3/2}=(r^2(cos^2(\theta)+sin^2(\theta)^2))^{3/2}=r^{6/2}=r^3
    \end{equation*}
\end{center}

On a donc:
\begin{center}
    \begin{equation*}
        \frac{\partial g}{\partial r}=3r^2
    \end{equation*}
\end{center}
Et:
\begin{center}
    \begin{equation*}
        \frac{\partial g}{\partial \theta}=0
\end{equation*}
\end{center}

On a:
\begin{center}
    \begin{equation*}
        \frac{\partial g}{\partial \theta}=-rsin(\theta)\frac{\partial f}{\partial x}(rcos(\theta),rsin(\theta))+rcos(\theta)\frac{\partial f}{\partial y}(rcos(\theta),rsin(\theta))
\end{equation*}
\end{center}

Et:
\begin{center}
    \begin{equation*}
        \frac{\partial g}{\partial r}=cos(\theta)\frac{\partial f}{\partial x}(rcos(\theta),rsin(\theta))+sin(\theta)\frac{\partial f}{\partial y}(rcos(\theta),rsin(\theta))
\end{equation*}
\end{center}

En utilisant la même méthode qu'en cours dans l'exercice A.2.15 en composant ces deux equations, on peut déterminer que:
\begin{center}
    \begin{equation*}
        \frac{\partial f}{\partial x}=cos(\theta)\frac{\partial g}{\partial r}(rcos(\theta),rsin(\theta))-\frac{sin(\theta)}{r}\frac{\partial g}{\partial \theta}(rcos(\theta),rsin(\theta))
    \end{equation*}
\end{center}
Or, on sait que \begin{math}\frac{\partial g}{\partial \theta}=0
\end{math}et: \begin{math}
    \frac{\partial g}{\partial r}=3r^2
\end{math}

Donc:
\begin{center}
    \begin{equation*}
        \frac{\partial f}{\partial x}(rcos(\theta),rsin(\theta))=g_1(r,\theta)=3r^2cos(\theta)
    \end{equation*}
\end{center}

b. On a donc:
\begin{center}
    \begin{equation*}
        \frac{\partial g_1}{\partial r}=6rcos(\theta)
    \end{equation*}
\end{center}
Et:
\begin{center}
    \begin{equation*}
        \frac{\partial g_1}{\partial \theta}=-3r^2sin(\theta)
    \end{equation*}
\end{center}
\end{document}